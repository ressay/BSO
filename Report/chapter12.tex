\chapter*{Conclusion générale}
Arrivé à la fin de ce projet, et après beaucoup de temps passé à apprendre, modifier et tester les différents algorithmes vu en cours, nous pensons avoir achevé un travail que nous jugeons assez complet, nous avons exploré différents aspects de la résolution de problème, en partant des méthodes basique aux méthode plus avancées, nous pouvons résumé notre travail aux points suivants : 
\begin{itemize}
	\item Les méthodes classiques du début de l'ère de l'intelligence artificielle ont prouvé leur efficacité jusqu'au jour d'aujourd’hui, cependant leurs limite s'est vu apparaître en même temps que l'apparition de problèmes beaucoup plus complexes et surtout plus volumineux.
	\item De nouvelles méthodes ont fait leur apparition, sacrifiant le désir de trouver une solution exacte(out optimale) qui peut prendre un temps inconcevable pour être détermine, au profit de solutions, certe moins optimales mais qui demeurent une alternative raisonnable.
	\item Malgré le coté aléatoire et probabiliste des nouvelles approches méta-heuristique, leur façon de fonctionner en fait une représentation fidèle de la vie réelle en générale.
	\item Les meilleures méthodes classique ne sont pas encore a jeté à la poubelle, car elles peuvent encore êtres utilisées pour améliorer les méthodes modernes, à l'instar de ACO qui a su marier méthodes évolutionnaire et constructives
\end{itemize}